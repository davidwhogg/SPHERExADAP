\documentclass{article}

% page layout for NASA
\usepackage[letterpaper, textwidth=6.5in, textheight=9.0in]{geometry}
\usepackage{setspace}
\setstretch{1.08}
\usepackage{fancyhdr}
\setlength{\topmargin}{0.0in}
\setlength{\headheight}{3.0ex}\addtolength{\textheight}{-\headheight} % make room for headers
\setlength{\headsep}{3.0ex}\addtolength{\textheight}{-\headsep}
\pagestyle{fancy}
%\renewcommand{\headrulewidth}{0.0pt}
\fancyhf{}
\fancyhead[L]{\sffamily Hogg / Improved precision and new cosmological observables for \textsl{SPHEREx}}
\newcommand{\paragraphskip}{\smallskip}
\renewcommand{\paragraph}[1]{\par\paragraphskip\noindent\textbf{#1}}
\renewcommand{\subparagraph}[1]{\par\paragraphskip\noindent\textit{#1}}
\sloppy\sloppypar\raggedbottom\frenchspacing

% stuff for boxes and figures
\usepackage[framemethod=TikZ]{mdframed}
\newcounter{box}
\newcommand{\boxname}{Box}
\newcommand{\boxref}[1]{\boxname~\ref{#1}}
\newenvironment{explainer}{\begin{figure}[t!]\begin{mdframed}[roundcorner=2pt, backgroundcolor=black!05]\refstepcounter{box}}{\end{mdframed}\end{figure}}

% math macros
\newcommand{\unit}[1]{\mathrm{#1}}
\newcommand{\arcsec}{\unit{arcsec}}

\begin{document}

\paragraph{Project summary:}
The new \textsl{SPHEREx} Mission is making measurements of a kind that astrophysicists we have not had previously.
The spacecraft delivers a low-resolution visible-to-infrared spectrum of every position on the celestial sphere, with $6\,\arcsec$ pixels.
It produces this remarkable data set with no moving parts, making use of multiple detectors, all with ramped, narrowband filters.
The observing strategy is a calibrator's dream, as it moves, over time, every source on the sky onto every point in the detector space multiple times throughout a two-year mission, (impressively) making the data public as it goes.

Here we propose to do four-ish things with the \textsl{SPHEREx} data:

\paragraph{Reformat the data} into a publicly available, high-performance, column-oriented data format with an interface that permits very fast queries as a function of time, wavelength, detector position, or position on the celestial sphere.

\paragraph{Separate components} of the measured intensities using a geometry-informed causal separation technique.
This will improve the separation of the measured intensity field into contributions at the spacecraft, contributions from Zodiacal light, and contributions from outside the Solar System.

\paragraph{Develop and measure new cosmological observables} that correspond to marked cross-correlation functions between the intensity field and galaxies, in real space. These can be seen as very sensitive, differential ``stacking'' methods.

\paragraph{Interpret these observables} in terms of both traditional astrophysical signals and new physics signals from dark-matter--photon interactions.
This will create novel measurements of total (clustered) cosmic stellar mass and star-formation rate
It will also put sensitive and novel limits on new physics in the dark sector, in the form of dark-matter--photon interactions, which could come from scattering or interactions with light degrees of freedom.

\paragraphskip
These four project components will lead to deliverables in the form of open-source code, publicly available data interfaces, contributions to scientific meetings, and refereed scientific papers.

\clearpage
\section{New cosmological observables}
If you have many measurements of \emph{specific intensity} $I_\nu$ (see \boxref{box:intensity} on page~\pageref{box:intensity} for the definition), untargeted, all over the celestial sphere, and many observations of \emph{galaxies} with measured angular positions on the sphere and measured redshifts, many new cosmological observables become accessible.
\begin{explainer}
    \paragraph{\boxname~\thebox\ --- A note on specific intensity:
    \label{box:intensity}}
\end{explainer}

\section{Reformatting the \textsl{SPHEREx} data}

\section{Causal separation of components}

\section{The galaxy--photon cross-correlation function}
\begin{explainer}
    \paragraph{\boxname~\thebox\ --- A note on the marked cross-correlation function:}
\end{explainer}

\section{New cosmological measurements}

\section{New searches for new physics}
\begin{explainer}
    \paragraph{\boxname~\thebox\ --- A note on light degrees of freedom in the dark sector:}
\end{explainer}

\section{Management and timeline}

\section{Prior NASA support}

\section{Student mentoring plan}

\section{Data management plan}

\end{document}
